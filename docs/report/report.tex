\documentclass[10pt, a4paper, draft]{article}
\usepackage[ngerman]{babel}
\usepackage[utf8]{inputenc}
\usepackage[T1]{fontenc}
\usepackage{color}
\usepackage{hyperref}
\usepackage{comment}
\usepackage[backend=biber,style=alphabetic,]{biblatex}

\addbibresource{literature.bib}

\title{Projektbreicht Smart Music Player}
\author{Anton Bracke\\Jan Eberlein\\Tom Calvin Haak\\Julian Hahn\\Nick Loewecke}

\begin{document}
\maketitle
\newpage
\tableofcontents
\newpage

\section{Einleitung}
\subsection{Unternehmen}
\colorbox{red}{TODO:}was macht macio aus
\subsection{Projektidee}
Im Rahmen des Projekt Informatik möchte Macio ihr Portfolio im IoT-Bereich erweitern, sowie ihren Empfangsraum im Standort Kiel verschönern.
Hierfür soll eine smarte Spielzeug-Box gebaut werden.
Smarte Spielzeuge gibt es im kommerziellen Bereich viele, daher soll dieses Projekt eine Open-Source-Alternative schaffen.\\
Genauer handelt es sich um eine Musik-Box, die NFC-Chips lesen und Spotify Connect unterstützen soll.
Auf die Box können dann Spielzeuge (z.B. in Form von kleinen Figuren) mit integrierten NFC-Chips gestellt werden, um spezifische Musik abspielen zu lassen.
Die Musik wird von Spotify-Connect über eine bereits bestehende Musik-Anlage abgespielt.
Falls es im Rahmen des Projektes möglich ist, sollen die Nutzer in der Lage sein, zwischen verschiedenen Musikanbietern zu wechseln.
NFC-Chips und die zugehörige Musik sollen über ein Web-basierte Benutzeroberfläche konfiguriert werden können.
Diese Benutzeroberfläche soll von der Box ausgeliefert und primär für Smartphone-Bedienung gestaltet werden.
Da es sich um ein Open Source Projekt mit entsprechender Lizenz handelt, muss auch eine aussagekräftige, öffentliche Dokumentation verfasst werden.
Macio stellt die benötigte Hardware zur Verfügung und unterstützt bei technischen Fragen.

\subsubsection{Minimal Requirements}
\begin{enumerate}
  \item NFC-Tags lesen, schreiben und entschlüsseln
  \item Mit Spotify Connect verbinden und arbeiten
  \item Responsive UI konzeptionieren und umsetzen
  \item Aussagekräftige Dokumentation mit Benutzerhandbuch
\end{enumerate}
\subsubsection{Stretch Goals}
\begin{enumerate}
  \item Sound Wiedergabe auf der Box selbst
  \item Unterstützung anderer Musikdienste / Plugin-Subsystem
  \item 3D-Modellierung und Print einer passenden Box
  \item Cloud-Anbindung der Box, Auslieferung des UI aus der Cloud
\end{enumerate}

\section{Projektplanung}

\subsection{Projekt Management}
\colorbox{red}{TODO:}ticket pool in github, gemeinsame Absprache (im team und mit Kunden) was in den aktuellen Milestone kommen soll, alles zentral, 3x daily standup, gemeinsamer google calendar, gemeinsamer Discord Server (für spontaneres und informelleres gemeinsames Arbeiten), interne team evaluation und regelmäßiges feedback

\subsection{Projektrisiken}
\colorbox{red}{TODO:}Welche Riskiken hatten wir + Risikomatrix? lohnt sich dieser Abschnitt?

\subsection{Phasenplanung}
\subsubsection{Projektstartphase}
\colorbox{red}{TODO:}Github aufsetzen, Tickets befüllen, Anforderungen verstehen, Team einarbeiten in Technologien

\subsubsection{Realisierungsphase}
\colorbox{red}{TODO:}Iteratives Arbeiten in Meilensteinen/Sprint mit Länge ca. 2 Wochen. Coden etc. Jedes Arbeitspaket wird einzeln getestet und von mindestens zwei anderen Personen reviewt (Technische Funktionalität wird automatisch getestet)

\subsubsection{Vermarktungsphase}
\colorbox{red}{TODO:}Testsachen aus Code nehmen, Anleitung schreiben, Informationen veröffentlichen

\subsection{Projektstrukturplan}
\colorbox{red}{TODO:}einzelne subsubsections über frontend, backend, hardware, devops, und dann erklären was das beinhaltet?

\subsection{Beispielhafte Arbeitspakete}
\colorbox{red}{TODO:}1-2-3 Beispielhafte Arbeitspakete ausarbeiten? Am besten auch welche, die von anderen abhängig

\subsection{Netzplan}
\colorbox{red}{TODO:}wie hängt alles zusammen mit approx Zeit, wobei wir die Zeit vielleicht weglassen?

\section{Machbarkeitsstudie}

\subsection{NFC Tag}
\subsubsection{lesen}
Um mit NFC Tags arbeiten zu können, müssen diese auch entschlüsselt bzw. gelesen werden können.
Hierfür ist ein Hardware NFC-Reader notwendig, der die Daten ausliest und an die Box kommuniziert.
Dieser emuliert dafür Keyboard Eingaben, die die Tag-IDs darstellen.
\textit{Evdev}, ein Kernel Modul von Linux, könnte dann zum Abgreifen dieser Keyboard-Eingaben genutzt werden. Mit \textit{node-evdev}\footnote{https://github.com/sdumetz/node-evdev} kann \textit{evdev} auch mit Node genutzt werden. Mit dem Fork\footnote{https://github.com/anbraten/node-evdev} von Anbraten wird zusätzlich der Raspberry Pi und die Typescript Unterstützung zur Verfügung gestellt.

\subsubsection{schreiben}
Ein NFC-Tag hat generell eine feste ID.
Um weitere Daten auf einen NFC-Tag schreiben zu können, benötigt der NFC-Tag also einen eigenen Speicher.
Ist dieser vorhanden, können dort z.b. Kontaktdaten hinterlegt werden. Werden diese dann von einem Smartphone gelesen, öffnet sich die Kontakte-App und der auf dem NFC-Tag gespeicherte Kontakt kann abgespeichert werden.
Dafür wäre einerseits ein spezieller NFC-Reader, der auch schreiben kann, sowie eine spezielle Library notwendig.

\subsection{Raspberri Pi}
\subsubsection{Docker Integration}
Auf einem Raspberri Pi Docker zu installieren und zum Laufen zu bringen wird in vielen Anleitungen online beschrieben\footnote{https://phoenixnap.com/kb/docker-on-raspberry-pi}.

\subsubsection{Öffentlich zugängliches Web Interface}
Auf einem Raspberri Pi könnte eine Web Anwendung gehosted werden, welche die allgemeine Web Anwendung für alle Boxen darstellen soll.
Um diese Web Anwendung von außerhalb des eigenen Netzwerkes erreichen zu können, muss innerhalb des Routers ein Port ge-forwarded werden.
Dann kann der Pi und dessen Webinterface unter der öffentlichen IP xxx.xxx.xxx.xxx des Routers erreicht werden.
Da sich die öffentliche IP-Adresse eines privaten Internet-Anschlusses in der Regel täglich ändert, wird zum einfachen finden der IP ein DynDns Service benötig, welcher eine feste Domain in die wechselnde IP Adresse des Routers übersetzt.
Alternativ ginge es auch ohne Port-Forwarding mit nginx und ngrok \footnote{https://vatsalyagoel.com/setting-up-a-public-web-server-using-a-raspberry-pi-3/}.
Für Unerfahrene wären diese notwendingen Schritte zu Beginn etwas komplexere Thematiken. Der effektive Arbeitsaufwand hängt daher auch sehr stark von der Erfahrung der einzelnen Teammitglieder ab.

\subsubsection{URL für UI festlegen}
Über ein mDNS Service, der auf dem Raspberri Pi läuft, wäre es möglich für die statische Public-IP eine eigene URL anzulegen.
Dafür sind verschiedene mDNS Services möglich, potentiell ist auch eine Domain notwendig.

\subsection{User Interface}
\subsubsection{Zugriff auf NFC Reader von Cloud Anwendung}
Um von der App auf die Daten vom NFC-Reader der verschiedenen Boxen zuzugreifen, wäre ein zentrales Backend mit einer API sinnvoll.
Der Computer der Box könnte beim Lesen eines NFC-Tag Daten über einen API Call an die Cloud Anwendung schicken, sodass die zusätzlich zu dem Backend verbundene App das gewünschte Event triggern kann.

\subsubsection{Login via Spotify, Youtube, etc.}
Es gibt ein Feathers Plugin, welches die Möglichkeit bietet OAuth Provider zu nutzen, um sich über andere Services wie Spotify anzumelden.

\subsubsection{Gleichen Nutzer bei verschiedenen Loginvarianten wiedererkennen}
Um gleiche Nutzer zu erkennen, müssten Merkmale angelegt werden, über die diese Nutzer wiedererkennbar wären.
Die E-Mail wäre hierbei ein geeignetes Merkmal, da dies einzigartig ist. Über gesetzte Scopes in der OAuth Anfrage kann diese vom jeweiligen Provider mitgeliefert werden.
Um die E-Mail als Wiedererkennungsmerkmal zu verwenden, muss vorausgesetzt sein, dass Nutzer immer die gleiche E-Mail bei den unterschiedlichen Providern nutzen. Dies ist aber nicht immer der Fall.
Daher könnte dem (bereits eingeloggten) Nutzer die Möglichkeit gegeben werden, weitere Accounts zu dem bestehenden hinzuzufügen und entsprechend in der Datenbank zu hinterlegen.

\subsubsection{Musik Artwork laden}
Sollte bei der Verwendung von Spotify kein Problem sein, da zu jeder Anfrage von Titeln oder Liedern auch eine Liste von Bildern enthalten ist.\footnote{https://stackoverflow.com/questions/10123804/retrieve-cover-artwork-using-spotify-api}

\subsubsection{Eigene Bilder hochladen}
Eigene Bilder hochzuladen sollte möglich sein. In unserem Kontext mit Vue.js und Node.js würde das Plugin \textit{vue-picture-input} helfen.
Mit einem Axios Post könnte das Bild an das Backend gesendet werden. \footnote{https://www.digitalocean.com/community/tutorials/vuejs-uploading-vue-picture-input}

\subsubsection{Spotify Connect Lautsprecher auswählen}
Das Auswählen von einem spezifischen Spotify Connect Lautsprecher ist möglich.
Über einen API Call an die Spotify API mit dem Endpunkt \textit{/v1""/me""/player""/device} wird eine Liste von allen verbundenen Geräten geliefert. Über den Endpunkt kann ein entsprechendes Lied zum Abspielen über den jeweiligen \textit{Spotify Connected Speaker} übergeben werden.
Sollte nicht explizit ein Lautsprecher angegeben werden, so wird der zuletzt aktive genutzt. Dieser hat bei \textit{is\_active} den Wert \textit{true}. \footnote{https://developer.spotify.com/documentation/web-api/guides/using-connect-web-api/}

\subsubsection{Spotify Connect Lautsprecher speichern}
Die Liste von verbundenen Geräten, die über einen Call an die Spotify API erhalten wird, enthält auch ein eindeutiges Feld \textit{id}, welches sich zusammen mit einem Namen speichern lässt.

\subsubsection{In der Cloud Anwendung die eigene Box auswählen / verbinden}
Bei der Ersteinrichtung könnte der Nutzer über die Eingabe der MAC Adresse oder über eine andere festgelegte ID die eigene Box finden und zu seinem Account hinzufügen. Die jeweilige Box wäre dem System anschließend bekannt und könnte zum Beispiel über den vom Nutzer gewählten Namen wiedergefunden und ausgewählt werden.

\subsubsection{Boxdaten über Cloud Anwendung ändern}
Um die auf der Box gespeicherten Daten aus der Cloud Anwendung heraus zu ändern, könnte ein direkter Aufruf einer API, welche auf der eigenen Box läuft, genutzt werden. Um die Verbindung zu der Box aufbauen zu können, könnte diese auf dem Cloud Backend die entsprechenden Verbindungsdaten hinterlegen.
\subsubsection{Unterstützung von Youtube Music}
Eine Umsetzung könnte sich als umständlich erweisen, da es bisher noch keine dedizierte Youtube Music API gibt.

\subsubsection{Unterstützung von Youtube}
Youtube bietet die Möglichkeit, nach Videos zu suchen \footnote{https://developers.google.com/youtube/v3/}. Um diese Videos auf dem Raspberry als Musik abzuspielen würde sich \textit{youtube-dl} zum downloaden der Videos als \textit{.mp3} Dateien und \textit{omxplayer} zum Abspielen anbieten.
Hierfür wäre allerdings ein entsprechender Lautsprecher am Raspberry erforderlich. Ein vergleichbares System zu Spotify Connect existiert derzeit noch nicht.

\subsubsection{Unterstützung von Apple Music}
Apple Music bietet hier mit deren MusicKit JS\footnote{https://developer.apple.com/documentation/musickitjs/} eine Möglichkeit, um Musik abzuspielen.

\subsubsection{Unterstützung von Deezer}
Deezer lässt sich vergleichbar zu Spotify über eine API steuern.\footnote{http://developers.deezer.com/login?redirect=/api}

\subsubsection{Unterstützung von eigener Musik (USB Stick, MicroSD Karte, Cloud)}
Da hier extrem viele Möglichkeiten mit verschiedensten Problemen existieren, wird dieser Punkt vorest vernachlässigt.

\subsection{Sonstiges}
\subsubsection{3D Print version}
Da unsere Box nicht übermäßig groß sein soll, müssten handelsübliche 3D-Drucker von der Größe ausreichend sein.
Das Modellieren einer 3D-Print Version ist am Ende von der Expertise der Gruppe abhängig.
Abgesehen davon sollte es kein besonderes Problem darstellen.

\subsubsection{Sound Wiedergabe auf der Box selbst}
Manche Pi Modelle verfügen über einen On-Board Audio Anschluss.
Die Wiedergabe über diesen ist qualitativ für Musik meist ungeeignet und sollte daher über ein weiteres Audiomodul oder eine externe Soundkarte erfolgen.
Innerhalb der Raspberri Pi Reihe gibt es dafür Accessoires, die circa 20-30€ kosten.\footnote{https://www.raspberrypi.org/products/}.
Zur Wiedergabe auf der Box selbst müsste dafür auf dem Raspberry eine Spotify Instanz laufen, damit auch diese als Connected Speaker erkannt wird.
Hierfür existieren Libraries wie \textit{raspotify}\footnote{https://github.com/dtcooper/raspotify}.

\subsubsection{Box unter 30€ Kosten}
Mit einem Raspberri Pi wäre dieses Ziel möglich, es könnte aber kein Pi ab Model 3 verwendet werden, da diese über dem Ziel liegen.
Mit dem Raspberri Pi Zero W mit eingebautem W-Lan und einem USB Port für den NFC-Reader gäbe es ein kostengünstiges Model, welches für ca. 10\$ erhältlich ist \footnote{https://www.raspberrypi.org/products/raspberry-pi-zero-w/}.

\section{Design Mockups}
\colorbox{red}{TODO:}setze pdfs ein

\section{Durchführung}

\subsection{Technologien und Hilfsmittel}
\colorbox{red}{TODO:}Vue, vscode, devops krams, etc
Entwickelt wird mit Visual Studio Code, da es eine einfache Nutzung des Linux-Subsystems ermöglicht. \footnote{https://code.visualstudio.com/docs/remote/wsl}.

\subsection{Deployment Cycle}
\colorbox{red}{TODO:} ziehe ticket > assigne dich selbst > draft PR > wenn fertig, setzte \textit{undraft} > assigne 2 reviewer > merge master

\subsection{Probleme während der Durchführung}
\colorbox{red}{TODO:}




\section{Projekt Meilensteine}

\subsection{Meilenstein 1}
Die Laufzeit des ersten Meilensteins erstreckte sich vom 29.10.2020 bis zum 11.11.2020 und war auf die Planung, Struktur und das Vorgehen des Projektes fokusiert.
\subsubsection{Ziel}
Zum Ende des Meilensteins sollte die Struktur und das Grundgerüst für das Projekt fertiggestellt sein.
Hierfür sollte das Konzept der Software-Architektur sowie die genutzten Technologien (später Technologie-Stack) auf Basis einer Anforderungsanalyse mit dem Kunden festgelegt werden.
Außerdem sollten die ersten Tickets in das Backlog eingetragen werden.
Auch musste das Projekt dahingehend geplant werden, dass jedem Teammitglied durchgehend zu bearbeitende Arbeitspakete zur Verfügung standen.
Um ein schnellen und reibungslosen Start für die Entwicklung zu gewährleisten, sollte das Repository entsprechend vorbereitet werden.
\subsubsection{Probleme}
Das Team musste das Projekt so entsprechend planen, dass jedes Teammitglied durchgehend sinnvoll arbeiten kann.
\subsubsection{Lösungen}
Durch bereits vorhandene Erfahrung mit verschiedenen Technologie-Stacks und ähnlichen Problemen, wurde sich auf den Technologie-Stack geeinigt
Die verwendeten Technologien sind dem Abschnitt 5.1 zu entnehmen. \colorbox{red}{Vor Abgabe kontrollieren}
Um für jedes Teammitglieds ein sinnvolles kontinuierliches Arbeiten zu gewährleisten, einigte sich das Team auf einen ständig gefüllten Ticket-Pool im Repository, sodass jederzeit selbstständig eine Aufgabe gefunden und bearbeitet werden konnte.
\subsubsection{Product Increment}
Das Ergebnis des ersten Meilensteins bestand, abweichend von den späteren, nicht in einem Produkt Inkrement, sondern legte die Grundbausteine für die Entwicklung und das Teamwork.
Neben dem bereits erwähnten Technologie-Stack, wurde als Host für die Applikation ein Raspberry Pi gewählt.
Für die Entwicklung wurde ein Mono-Repository mit \textit{lerna}\footnote{https://lerna.js.org/} aufgesetzt, welches Front- und Backend zusammenfasste.
Durch die Anforderung der Veröffentlichung als Open Source Projekt wurde eine verständliche und hilfreiche Dokumentation nötig, die einem Benutzer die Installation und Verwendung der \textit{leek-box} aufzeigt.
So wurde dem Repository eine einleitende Readme und ein \glqq How to Contribute\grqq{} Guide hinzugefügt.
\subsubsection{Retrospektive}
Nach dem ersten Meilenstein fand noch keine nennenswerte Retrospektive statt.

\subsection{Meilenstein 2}
Der zweite Meilenstein wurde für den Zeitraum vom 11.11.2020 bis zum 25.11.2020 angesetzt.
Ziel war es, die ersten Grundbausteine für das Projekt legen.
\subsubsection{Ziel}
Damit das Team ein generelles Verständnis des Technologie-Stacks entwickeln konnte, sollte jedes Teammitglied eine \glqq Hello-World\grqq{} Übungsaufgabe absolvieren.
Diese beinhaltete das Anlegen eines Feathers-Services für ein virtuelles Haustier im Backend und die Darstellung im Frontend.
Darüber hinaus sollte sich mit dem Raspberry Pi und dem NFC-Reader auseinandergesetzt werden und die Möglichkeit NFC-Tags auszulesen implementiert werden.
Damit die Entwicklung zeitnah starten konnte, sollten die ersten Mockups für das User Interface erstellt werden sowie die Grundstruktur für den Bericht zu dem Projekt angefertigt werden, damit dieser Projektbegleitend aufgebaut werden konnte.
\subsubsection{Probleme}
Zu Beginn ging der im ersten Review geäußerte Kundenwunsch, bereits NFC-Tags auszulesen und anzeigen zu können, unter und stellte so kurzzeitig das rechtzeitige Abschließen der für den Milestone vorgesehenen Tickets in Frage.
Außerdem bereitete die Einarbeitung in die genutzten Technologien einigen Teammitgliedern einige Schwierigkeiten.
\subsubsection{Lösungen}
Bereits von Anfang an zeichnete sich durch die hohe Hilfsbereitschaft den anderen Teammitgliedern gegenüber ein hoher Zusammenhalt im Team ab.
Um das vom Kunden gewünschte Feature noch umzusetzen, wurden gegen Ende des Sprints Überstunden geleistet.
\subsubsection{Product Increment}
Das Produkt wurde in diesem Meilenstein sinnvoll um ein erstes User Interface erweitert, welches die Tag-ID vom angelegten NFC-Tags anzeigen konnte.
Der NFC-Reader wurde mit dem Raspberri Pi verbunden und als ein auslieferbares Docker Image bereitgestellt.
Auch wurden für die Entwicklung automatisierte Tests erstellt, welche neuen Code auf etablierte Code-Konventionen und Lauffähigkeit überprüften.
\subsubsection{Retrospektive}
Vor allem der Einsatz und das Know-How vom Teammitglied Anton wurde hier wertgeschätzt, welcher sich um die Automatisierungen von Tests und Deployment kümmerte.
Auch das Gruppenklima wurde vom Team sehr gelobt, die Zusammenarbeit, Kommunikation und die regelmäßigen Meetings waren liefen sehr gut.
Neu gelernt hat das Team bei diesem Meilenstein vor allem wie Vue, Feathers und Tailwind funktionieren.
Darüber hinaus kamen viele das erste mal Automatisierungsprozessen, Code-Reviews und Pull Request in Kontakt.
Das Team bemängelte bei dem Meilenstein aber auch, dass der Kundenwunsch zu spät behandelt wurde und die Prioritäten falsch gesetzt wurden.
Somit wurde als Verbesserungen für die nächsten Meilensteine eine bessere Koordination und Aufgabenverteilung vorgeschlagen.

\subsection{Meilenstein 3}
Vom 25.11.2020 bis zum 17.12.2020 fand die Umsetzung von Meilenstein drei statt.
Die benötigten Anwendungsszenarien und Mockups sollten in diesem Sprint finalisiert werden.
Außerdem sollte der vom Kunden geäußerte Wunsch, dass beim Einlesen eines NFC-Tags die entsprechend hinterlegte Musik abgespielt wird umgesetzt werden.
\subsubsection{Ziel}
Dafür musste im Backend ein NFC-Tag mit der jeweiligen Musik angelegt werden.
Um die hohen Anforderungen an die Usability umsetzten zu können, musste sich damit beschäftigt werden, wie Nutzer mit der Anwendung interagieren könnte,.
Daher sollten auch Sequenzdiagramme und weitere Mockups für das User Interface erstellt werden.
Damit jedes Teammitglied auch ohne Hardware arbeiten konnte, sollte es auch möglich sein, das Einlesen von NFC-Tags zu emulieren.
\subsubsection{Probleme}
Aufgrund von vorherrschenden Abhängigkeiten mussten Teammitglieder teilweise auf die Vollendung anderer Tickets warten, bevor sie mit diesen begannen.
Ansonsten sind nur kleinere Problematiken, insbesondere bei den Design Mockups, aufgetreten.
Bedingt dadurch, dass das Team nur aus Informatikern bestand, gab es keine ausführlichen Erfahrungen mit Design und Usability Engineering.
\subsubsection{Lösungen}
Macio unterstützte das Team, durch die Vermittlung eines Designers ihrer Firma, der die vom Team angefertigten Mockups überarbeitete.
Das Erstellen der Sequenzdiagramme half bei dem Team dabei, das Verhalten eines Nutzers besser einzuschätzen und das User Interface entsprechend zu konzipieren.
\subsubsection{Product Increment}
Zusätzlich zu dem Wunsch, beim Anlegen eines NFC-Tags die jeweilig hinterlegte Musik abzuspielen, war auch das Einloggen in einen bestehenden Spotify Account möglich.
Außerdem wurden auf der Startseite die NFC-Tags entsprechend der zugeordneten Gruppe angezeigt.
Jegliche Neuerung auf dem Master wurden nun auch auf Github.io veröffentlicht.
\subsubsection{Retrospektive}
Deutlich besser lief in diesem Sprint die Problembewältigung durch Pair Programming, welches die Teamarbeit weiter verbesserte.
Auch der Workflow mit Github, den Pull Request sowie Code Reviews wurde mehr verinnerlicht.
Das Team lernte in diesem Sprint vor allem Vue Hooks kennen, und die Stärken von Feathers in Verbindung mit OAuth.
Festgestellt wurde außerdem, dass einige Tickets weiterhin stark voneinander abhängig waren, weshalb das Bearbeiten einiger Tickets nicht direkt möglich war.
Auf diese Problematik sollte weiterhin geachtete und nach Möglichkeit gegengesteuert werden.
Um das Risiko der Entstehung zukünftiger Probleme zu senken wurde festgelegt mögliche Problemstellungen nicht erst im Standup, sondern so früh wie möglich mitzuteilen.
Genauso sollte neu Gelerntes besser kommuniziert werden, um die Arbeit effizienter zu gestalten.

\subsection{Meilenstein 4}
Meilenstein vier lag innerhalb der Weihnachtszeit und wurde deswegen mit einem größeren Zeitraum vom 17.12.2020 bis 14.01.2021 geplant.
Mit diesem Meilenstein begann die funktionale und visuelle Umsetzung der Hauptfunktionen.
\subsubsection{Ziel}
Wie auch beim vorherigen Meilenstein fokussierte sich das Team auf die vom Kunden gewünschten Funktionalitäten wie das Anlegen und Bearbeiten von eigenen Tags.
Dazu gehörte:
\begin{enumerate}
  \item Scannen des Tags
  \item Vergeben eines eigenen Namens
  \item Auswählen von Musik
  \item Auswählen eines eigenen Bilds
\end{enumerate}
Auch sollten für den nächsten Schritt die Mockups erstellt werden, wie ein Nutzer den Lautsprecher zur Wiedergabe auswählt sowie seine eigene \textit{Leek Box} initial einrichtet.
\subsubsection{Probleme}
Zu diesem Meilenstein sind vermehrt größere Probleme aufgetreten.
Die tatsächliche Umsetzung und Planung/Einschätzung divergierte in diesem Meilenstein sehr stark.
Es wurde sich zu sehr in Details verloren, sodass der Fokus auf die wichtigen Tickets verloren ging.
Das führte dazu, dass zum Review viele halbfertige Tickets entweder nur schlecht getestet oder gar nicht im Master-Branch implementiert waren.
\subsubsection{Lösungen}
Kurzfristig wurde eine Team interne Evaluation der einzelnen Mitglieder durchgeführt.
Das half einerseits dabei, dass Probleme gelöst und die Teammotivation sowie die Produktivität stieg, als auch das generelle Teamklima weiter gebessert wurde.
\subsubsection{Product Increment}
Das Produkt wurde um die Grundfunktionen erweitert.
Es ist nun möglich, Tags anzulegen und entsprechend für diese Namen, Musik und ein Bild festzulegen.
Genauso ist es dem Nutzer nun möglich, alle angelegten Tags zu durchsuchen.
Es wurde weiterhin auch das Design Mockup von dem von macio gestellten Designer in Vue größtenteils umgesetzt.
Gleichzeitig wurden hierbei die Komponenten zur einfacheren Verwendung als möglichst allgemeine Vue Komponenten angelegt, um ein einheitliches Design zu ermöglichen.
Für die Entwickler wurden auch die automatisierten Tests um einen Test auf Typsicherheit erweitert.
\subsubsection{Retrospektive}
Besonders gut gefiel dem Team das Teamklima, vor allem des ehrliche Miteinander.
Trotz Komplikationen und Probleme konnte dem Kunden am Ende ein funktionierendes Product Increment geliefert werden.
Pull Requests und Tickets wurden besser beschrieben, sodass Probleme verständlicher waren.
Vor allem durch das interne Team Review wurde die Zusammenarbeit weiter verbessert.
Das Team konnte das Wissen zu Vue, vor allem die Themen Reaktivität, Events und Kommunikation zwischen Komponenten ausbauen.
Stark bemängelt wurde der Scope des Meilensteins.
Das Zeitmanagement war an dieser Stelle nicht gut, es wurde sich zu viel vorgenommen und man hatte am Ende zu wenig Zeit.
Das Team stimmte hier überein, dass eine bessere Planung sowie Abstimmung für die nächsten Meilensteine notwendig wäre.
Auch sollte eine frühere interne Deadline gesetzt werden, damit nicht kurz vor dem Sprintende noch viele Fehler auftreten und behoben werden müssen.
Um insgesamt auch die Arbeit kontinuierlicher zu gestalten, sollten Aufgaben in kleinere Probleme aufgeteilt werden.

\subsection{Meilenstein 5}
Der Meilenstein fünf vom 14.01.2021 bis zum 28.01.2021 sollte die Entwicklung am Produkt weitgehend zu abschließen, sodass sich das Team auf den Bericht fokussieren konnte.
\subsubsection{Ziel}
Geplant war, die letzten Anwendungszenarien final umzusetzen, sodass die letzten beiden Meilensteine höchstens zum Beheben kleinerer Fehler benötigt werden würden.
Somit war das Ziel die Umsetzung einer Oberfläche für die Einstellungen der \textit{Leek Box} selbst als auch der einzelnen Tags.
Die User Interface Elemente sollten, sofern noch nicht geschehen, ein einheitliches Design erhalten.
Auch sollte die Dokumentation zur Bedienung der \textit{Leek Box} fertiggestellt sein und die Arbeit für den Bericht begonnen werden.
\subsubsection{Probleme}
Wie beim vorherigen Meilenstein gestaltete sich auch hier das Zeitmanagement schwierig.
Gegen Ende fehlte es an Zeit, um neu umgesetzte Tickets auf Fehler zu testen und entsprechend fehlerfrei dem Kunden zu präsentieren.
\subsubsection{Lösungen}
Als Lösung für dieses Problem beschloss das Team dem Kunden im Review weniger neue Features vorzustellen als geplant.
\subsubsection{Product Increment}
Das Produkt Inkrement enthielt die implementierten Einstellungen der \textit{Leek Box}, die es dem Nutzer ermöglichen, den Wiedergabelautsprecher auswählen, die jeweilige Leek Box zu verwalten und sich von Spoitfy abzumelden.
Auch wurde ein Setup Guide angefertigt, sodass ein Nutzer selbstständig eine eigene \textit{Leek Box} einrichten könnte.
Bei dem Design wurden alle Unstimmigkeiten beseitigt.
\subsubsection{Retrospektive}
Insgesamt hat das Team sich an dieser Stelle vor allem über das sehr positive Feedback des Kunden gefreut.
Auch wurde die Produktivität des Teams sehr gelobt.
Das Team konnte außerdem weitere Vue Funktionen kennenlernen.
Problematisch war es, dass manche Tickets von mehreren gleichzeitig oder nacheinander bearbeitet wurden, sodass Arbeit teilweise doppelt erledigt wurde.
Außerdem erfolgte die Kommunikation von größeren Änderung an jedes Teammitglied teilweise nur lückenhaft.
Lösungsvorschläge waren Verbesserung des Zeitmanagements durch \colorbox{yellow}{???} und eine bessere Kommunikation durch Festhalten von größeren Änderungen in den Pull Requests und Markierung aller Gruppenmitglieder.

\subsection{Meilenstein 6}
Mit dem Meilenstein sechs vom 28.01.2021 bis zum 15.02.2021 sollte die Entwicklung beendet und der Fokus komplett auf den Bericht gelegt werden, sodass der nächste und letzte Meilenstein für die letzten Korrekturen im Bericht genutzt werden konnte.
\subsubsection{Ziel}
Neben der Fokussierung auf den Bericht sollte auch das User Interface auf einheitliches Design überprüft und die Entwicklung und Fehlerbehebung abgeschlossen werden.
\subsubsection{Probleme}
\colorbox{red}{TODO:}schlauchiger start in den report

\subsubsection{Lösungen}
\colorbox{red}{TODO:}Welche Lösungen haben wir gefunden?

\subsubsection{Product Increment}
Als letzte Änderungen am Produkt hat nun der Nutzer die Möglichkeit, verschiedene Leek Boxen auszuwählen bzw. neu eingerichtete Leek Boxen mit seinem Konto zu verbinden.
Die letzten Bugfixes wurden umgesetzt sowie wurden die letzten Details aus dem Design Mockup übernommen.
\subsubsection{Retrospektive}
\colorbox{red}{TODO:}Was haben wir dabei gelernt? Neue Erkentnisse? Neue Sichtweisen?
Was lief gut, neu gelernt, was lief nicht so gut, was verbessern?

\subsection{Meilenstein 7}
Der finale Meilenstein sieben vom 15.02.2021 bis zum 05.03.2021 sollte nur noch die letzten Korrekturen des Berichtes und Vorbereitung der Finalen Abgabe beinhalten.
\colorbox{red}{TODO:}Kurze Einleitung, von, bis
\subsubsection{Ziel}
\colorbox{red}{TODO:}was haben wir uns vorgenommen, was war das ziel, was wollten wir schaffen?
\subsubsection{Probleme}
\colorbox{red}{TODO:}welche prrobleme sind aufgetreten?

\subsubsection{Lösungen}
\colorbox{red}{TODO:}Welche Lösungen haben wir gefunden?

\subsubsection{Product Increment}
\colorbox{red}{TODO:}Was ist am Ende dabei rumgekommen?

\subsubsection{Retrospektive}
\colorbox{red}{TODO:}Was haben wir dabei gelernt? Neue Erkentnisse? Neue Sichtweisen?
Was lief gut, neu gelernt, was lief nicht so gut, was verbessern?

\section{Erkenntnisse}

\section{Code Walkthrough}
\colorbox{red}{TODO:}vielleicht interessante Code Snippets?

\section{Testing}
\colorbox{red}{TODO:}wie haben wir getestet, haben wir getestet?

\section{Technische Diagramme}
\colorbox{red}{TODO:}ER Diagramme, UML, solcher krams

\newpage
\section{Anhang}
\colorbox{red}{BEISPIEL, DELETE THIS} Buchreferenz \cite{Literaturbeispiel:tom} oder Seitenref \cite{google}
\printbibliography
\end{document}
