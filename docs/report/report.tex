\documentclass[10pt, a4paper, draft]{article}
\usepackage[ngerman]{babel}
\usepackage[utf8]{inputenc}
\usepackage[T1]{fontenc}
\usepackage{color}
\usepackage{hyperref}
\usepackage{comment}
\usepackage[backend=biber,style=alphabetic,]{biblatex}

\addbibresource{literature.bib}

\title{Projektbreicht Smart Music Player}
\author{Anton Bracke\\Jan Eberlein\\Tom Calvin Haak\\Julian Hahn\\Nick Loewecke}

\begin{document}
\maketitle
\newpage
\tableofcontents
\newpage

\section{Einleitung}
\subsection{Unternehmen}
\colorbox{red}{TODO:}was macht macio aus
\subsection{Projektidee}
Im Rahmen des Projekt Informatik möchte Macio ihr Portfolio im IoT-Bereich erweitern, sowie ihren Empfangsraum im Standort Kiel verschönern.
Hierfür soll eine smarte Spielzeug-Box gebaut werden.
Smarte Spielzeuge gibt es im kommerziellen Bereich viele, daher soll dieses Projekt eine Open-Source-Alternative schaffen.\\
Genauer handelt es sich um eine Musik-Box, die NFC-Chips lesen und Spotify Connect unterstützen soll.
Auf die Box können dann Spielzeuge (z.B. in Form von kleinen Figuren) mit integrierten NFC-Chips gestellt werden, um spezifische Musik abspielen zu lassen.
Die Musik wird von Spotify-Connect über eine bereits bestehende Musik-Anlage abgespielt.
Falls es im Rahmen des Projektes möglich ist, sollen die Nutzer in der Lage sein, zwischen verschiedenen Musikanbietern zu wechseln.
NFC-Chips und die zugehörige Musik sollen über ein Web-basierte Benutzeroberfläche konfiguriert werden können.
Diese Benutzeroberfläche soll von der Box ausgeliefert und primär für Smartphone-Bedienung gestaltet werden.
Da es sich um ein Open Source Projekt mit entsprechender Lizenz handelt, muss auch eine aussagekräftige, öffentliche Dokumentation verfasst werden.
Macio stellt die benötigte Hardware zur Verfügung und unterstützt bei technischen Fragen.

\subsubsection{Minimal Requirements}
\begin{enumerate}
  \item NFC-Tags lesen, schreiben und entschlüsseln
  \item Mit Spotify Connect verbinden und arbeiten
  \item Responsive UI konzeptionieren und umsetzen
  \item Aussagekräftige Dokumentation mit Benutzerhandbuch
\end{enumerate}
\subsubsection{Stretch Goals}
\begin{enumerate}
  \item Sound Wiedergabe auf der Box selbst
  \item Unterstützung anderer Musikdienste / Plugin-Subsystem
  \item 3D-Modellierung und Print einer passenden Box
  \item Cloud-Anbindung der Box, Auslieferung des UI aus der Cloud
\end{enumerate}

\section{Projektplanung}

\subsection{Projekt Management}
\colorbox{red}{TODO:}ticket pool in github, gemeinsame Absprache (im team und mit Kunden) was in den aktuellen Milestone kommen soll, alles zentral, 3x daily standup, gemeinsamer google calendar, gemeinsamer Discord Server (für spontaneres und informelleres gemeinsames Arbeiten), interne team evaluation und regelmäßiges feedback

\subsection{Projektrisiken}
\colorbox{red}{TODO:}Welche Riskiken hatten wir + Risikomatrix? lohnt sich dieser Abschnitt?

\subsection{Phasenplanung}
\subsubsection{Projektstartphase}
\colorbox{red}{TODO:}Github aufsetzen, Tickets befüllen, Anforderungen verstehen, Team einarbeiten in Technologien

\subsubsection{Realisierungsphase}
\colorbox{red}{TODO:}Iteratives Arbeiten in Meilensteinen/Sprint mit Länge ca. 2 Wochen. Coden etc. Jedes Arbeitspaket wird einzeln getestet und von mindestens zwei anderen Personen reviewt (Technische Funktionalität wird automatisch getestet)

\subsubsection{Vermarktungsphase}
\colorbox{red}{TODO:}Testsachen aus Code nehmen, Anleitung schreiben, Informationen veröffentlichen

\subsection{Projektstrukturplan}
\colorbox{red}{TODO:}einzelne subsubsections über frontend, backend, hardware, devops, und dann erklären was das beinhaltet?

\subsection{Beispielhafte Arbeitspakete}
\colorbox{red}{TODO:}1-2-3 Beispielhafte Arbeitspakete ausarbeiten? Am besten auch welche, die von anderen abhängig

\subsection{Netzplan}
\colorbox{red}{TODO:}wie hängt alles zusammen mit approx Zeit, wobei wir die Zeit vielleicht weglassen?

\section{Machbarkeitsstudie}

\subsection{NFC Tag}
\subsubsection{lesen}
Um mit NFC Tags arbeiten zu können, müssen diese auch entschlüsselt bzw. gelesen werden können.
Hierfür ist ein Hardware NFC-Reader notwendig, der die Daten ausliest und an die Box kommuniziert.
Dieser emuliert dafür Keyboard Eingaben, die die Tag-IDs darstellen.
\textit{Evdev}, ein Kernel Modul von Linux, könnte dann zum Abgreifen dieser Keyboard-Eingaben genutzt werden. Mit \textit{node-evdev}\footnote{https://github.com/sdumetz/node-evdev} kann \textit{evdev} auch mit Node genutzt werden. Mit dem Fork\footnote{https://github.com/anbraten/node-evdev} von Anbraten wird zusätzlich der Raspberry Pi und die Typescript Unterstützung zur Verfügung gestellt.

\subsubsection{schreiben}
Ein NFC-Tag hat generell eine feste ID.
Um weitere Daten auf einen NFC-Tag schreiben zu können, benötigt der NFC-Tag also einen eigenen Speicher.
Ist dieser vorhanden, können dort z.b. Kontaktdaten hinterlegt werden. Werden diese dann von einem Smartphone gelesen, öffnet sich die Kontakte-App und der auf dem NFC-Tag gespeicherte Kontakt kann abgespeichert werden.
Dafür wäre einerseits ein spezieller NFC-Reader, der auch schreiben kann, sowie eine spezielle Library notwendig.

\subsection{Raspberri Pi}
\subsubsection{Docker Integration}
Auf einem Raspberri Pi Docker zu installieren und zum Laufen zu bringen wird in vielen Anleitungen online beschrieben\footnote{https://phoenixnap.com/kb/docker-on-raspberry-pi}.

\subsubsection{Öffentlich zugängliches Web Interface}
Auf einem Raspberri Pi könnte eine Web Anwendung gehosted werden, welche die allgemeine Web Anwendung für alle Boxen darstellen soll.
Um diese Web Anwendung von außerhalb des eigenen Netzwerkes erreichen zu können, muss innerhalb des Routers ein Port ge-forwarded werden.
Dann kann der Pi und dessen Webinterface unter der öffentlichen IP xxx.xxx.xxx.xxx des Routers erreicht werden.
Da sich die öffentliche IP-Adresse eines privaten Internet-Anschlusses in der Regel täglich ändert, wird zum einfachen finden der IP ein DynDns Service benötig, welcher eine feste Domain in die wechselnde IP Adresse des Routers übersetzt.
Alternativ ginge es auch ohne Port-Forwarding mit nginx und ngrok \footnote{https://vatsalyagoel.com/setting-up-a-public-web-server-using-a-raspberry-pi-3/}.
Für Unerfahrene wären diese notwendingen Schritte zu Beginn etwas komplexere Thematiken. Der effektive Arbeitsaufwand hängt daher auch sehr stark von der Erfahrung der einzelnen Teammitglieder ab.

\subsubsection{URL für UI festlegen}
Über ein mDNS Service, der auf dem Raspberri Pi läuft, wäre es möglich für die statische Public-IP eine eigene URL anzulegen.
Dafür sind verschiedene mDNS Services möglich, potentiell ist auch eine Domain notwendig.

\subsection{User Interface}
\subsubsection{Zugriff auf NFC Reader von Cloud Anwendung}
Um von der App auf die Daten vom NFC-Reader der verschiedenen Boxen zuzugreifen, wäre ein zentrales Backend mit einer API sinnvoll.
Der Computer der Box könnte beim Lesen eines NFC-Tag Daten über einen API Call an die Cloud Anwendung schicken, sodass die zusätzlich zu dem Backend verbundene App das gewünschte Event triggern kann.

\subsubsection{Login via Spotify, Youtube, etc.}
Es gibt ein Feathers Plugin, welches die Möglichkeit bietet OAuth Provider zu nutzen, um sich über andere Services wie Spotify anzumelden.

\subsubsection{Gleichen Nutzer bei verschiedenen Loginvarianten wiedererkennen}
Um gleiche Nutzer zu erkennen, müssten Merkmale angelegt werden, über die diese Nutzer wiedererkennbar wären.
Die E-Mail wäre hierbei ein geeignetes Merkmal, da dies einzigartig ist. Über gesetzte Scopes in der OAuth Anfrage kann diese vom jeweiligen Provider mitgeliefert werden.
Um die E-Mail als Wiedererkennungsmerkmal zu verwenden, muss vorausgesetzt sein, dass Nutzer immer die gleiche E-Mail bei den unterschiedlichen Providern nutzen. Dies ist aber nicht immer der Fall.
Daher könnte dem (bereits eingeloggten) Nutzer die Möglichkeit gegeben werden, weitere Accounts zu dem bestehenden hinzuzufügen und entsprechend in der Datenbank zu hinterlegen.

\subsubsection{Musik Artwork laden}
Sollte bei der Verwendung von Spotify kein Problem sein, da zu jeder Anfrage von Titeln oder Liedern auch eine Liste von Bildern enthalten ist.\footnote{https://stackoverflow.com/questions/10123804/retrieve-cover-artwork-using-spotify-api}

\subsubsection{Eigene Bilder hochladen}
Eigene Bilder hochzuladen sollte möglich sein. In unserem Kontext mit Vue.js und Node.js würde das Plugin \textit{vue-picture-input} helfen.
Mit einem Axios Post könnte das Bild an das Backend gesendet werden. \footnote{https://www.digitalocean.com/community/tutorials/vuejs-uploading-vue-picture-input}

\subsubsection{Spotify Connect Lautsprecher auswählen}
Das Auswählen von einem spezifischen Spotify Connect Lautsprecher ist möglich.
Über einen API Call an die Spotify API mit dem Endpunkt \textit{/v1""/me""/player""/device} wird eine Liste von allen verbundenen Geräten geliefert. Über den Endpunkt kann ein entsprechendes Lied zum Abspielen über den jeweiligen \textit{Spotify Connected Speaker} übergeben werden.
Sollte nicht explizit ein Lautsprecher angegeben werden, so wird der zuletzt aktive genutzt. Dieser hat bei \textit{is\_active} den Wert \textit{true}. \footnote{https://developer.spotify.com/documentation/web-api/guides/using-connect-web-api/}

\subsubsection{Spotify Connect Lautsprecher speichern}
Die Liste von verbundenen Geräten, die über einen Call an die Spotify API erhalten wird, enthält auch ein eindeutiges Feld \textit{id}, welches sich zusammen mit einem Namen speichern lässt.

\subsubsection{In der Cloud Anwendung die eigene Box auswählen / verbinden}
Bei der Ersteinrichtung könnte der Nutzer über die Eingabe der MAC Adresse oder über eine andere festgelegte ID die eigene Box finden und zu seinem Account hinzufügen. Die jeweilige Box wäre dem System anschließend bekannt und könnte zum Beispiel über den vom Nutzer gewählten Namen wiedergefunden und ausgewählt werden.

\subsubsection{Boxdaten über Cloud Anwendung ändern}
Um die auf der Box gespeicherten Daten aus der Cloud Anwendung heraus zu ändern, könnte ein direkter Aufruf einer API, welche auf der eigenen Box läuft, genutzt werden. Um die Verbindung zu der Box aufbauen zu können, könnte diese auf dem Cloud Backend die entsprechenden Verbindungsdaten hinterlegen.
\subsubsection{Unterstützung von Youtube Music}
Eine Umsetzung könnte sich als umständlich erweisen, da es bisher noch keine dedizierte Youtube Music API gibt.

\subsubsection{Unterstützung von Youtube}
Youtube bietet die Möglichkeit, nach Videos zu suchen \footnote{https://developers.google.com/youtube/v3/}. Um diese Videos auf dem Raspberry als Musik abzuspielen würde sich \textit{youtube-dl} zum downloaden der Videos als \textit{.mp3} Dateien und \textit{omxplayer} zum Abspielen anbieten.
Hierfür wäre allerdings ein entsprechender Lautsprecher am Raspberry erforderlich. Ein vergleichbares System zu Spotify Connect existiert derzeit noch nicht.

\subsubsection{Unterstützung von Apple Music}
Apple Music bietet hier mit deren MusicKit JS\footnote{https://developer.apple.com/documentation/musickitjs/} eine Möglichkeit, um Musik abzuspielen.

\subsubsection{Unterstützung von Deezer}
Deezer lässt sich vergleichbar zu Spotify über eine API steuern.\footnote{http://developers.deezer.com/login?redirect=/api}

\subsubsection{Unterstützung von eigener Musik (USB Stick, MicroSD Karte, Cloud)}
Da hier extrem viele Möglichkeiten mit verschiedensten Problemen existieren, wird dieser Punkt vorest vernachlässigt.

\subsection{Sonstiges}
\subsubsection{3D Print version}
Da unsere Box nicht übermäßig groß sein soll, müssten handelsübliche 3D-Drucker von der Größe ausreichend sein.
Das Modellieren einer 3D-Print Version ist am Ende von der Expertise der Gruppe abhängig.
Abgesehen davon sollte es kein besonderes Problem darstellen.

\subsubsection{Sound Wiedergabe auf der Box selbst}
Manche Pi Modelle verfügen über einen On-Board Audio Anschluss.
Die Wiedergabe über diesen ist qualitativ für Musik meist ungeeignet und sollte daher über ein weiteres Audiomodul oder eine externe Soundkarte erfolgen.
Innerhalb der Raspberri Pi Reihe gibt es dafür Accessoires, die circa 20-30€ kosten.\footnote{https://www.raspberrypi.org/products/}.
Zur Wiedergabe auf der Box selbst müsste dafür auf dem Raspberry eine Spotify Instanz laufen, damit auch diese als Connected Speaker erkannt wird.
Hierfür existieren Libraries wie \textit{raspotify}\footnote{https://github.com/dtcooper/raspotify}.

\subsubsection{Box unter 30€ Kosten}
Mit einem Raspberri Pi wäre dieses Ziel möglich, es könnte aber kein Pi ab Model 3 verwendet werden, da diese über dem Ziel liegen.
Mit dem Raspberri Pi Zero W mit eingebautem W-Lan und einem USB Port für den NFC-Reader gäbe es ein kostengünstiges Model, welches für ca. 10\$ erhältlich ist \footnote{https://www.raspberrypi.org/products/raspberry-pi-zero-w/}.

\section{Design Mockups}
\colorbox{red}{TODO:}setze pdfs ein

\section{Durchführung}

\subsection{Technologien und Hilfsmittel}
\colorbox{red}{TODO:}Vue, vscode, devops krams, etc
Entwickelt wird mit Visual Studio Code, da es eine einfache Nutzung des Linux-Subsystems ermöglicht. \footnote{https://code.visualstudio.com/docs/remote/wsl}.

\subsection{Deployment Cycle}
\colorbox{red}{TODO:} ziehe ticket > assigne dich selbst > draft PR > wenn fertig, setzte \textit{undraft} > assigne 2 reviewer > merge master

\subsection{Probleme während der Durchführung}
\colorbox{red}{TODO:}




\section{Projekt Meilensteine}

\subsection{Meilenstein 1}
Der erste Meilenstein begann am 29.10.2020 und ging bis zum 11.11.2020. Für den Anfang des Projektes war hier die Planung, Struktur und das Vorgehen des Projektes der Fokus.
\subsubsection{Ziel}
Zum Ende des Meilensteins sollte die Struktur und das Grundgerüst für das Projekt fertig gestellt sein.
Das umfasste sowohl das Konzept der Software-Architektur, die genutzten Technologien (später Technologie-Stack), einer Anforderungsanalyse mit dem Kunden sowie die Befüllung des Backlogs mit den ersten Tickets.Auch musste sich das Team damit beschäftigen, wie es das Projekt durchführen wollte und welche Management- und Teamorganisations-Methodiken dafür angewendet werden sollten.
Um ein schnellen und reibungslosen  Start für die Entwicklung zu gewährleisten, sollte das Repository entsprechend vorbereitet werden.
\subsubsection{Probleme}
Die ersten Probleme waren die Auswahl des Technologie-Stacks sowie die Auswahl der benötigten Hardware.
Auch musste sich das Team das Projekt so entsprechend planen, dass jedes Teammitglied durchgehend sinnvoll arbeiten kann.
\subsubsection{Lösungen}
Da bereits einige Teammitglieder Erfahrung mit verschiedenen Technologie-Stacks und ähnlichen Problemen hatten, wurde sich auf den Technologie-Stack geeinigt, bei dem die meiste Erfahrung im Team bestand.
Dieser Bestand aus den oben beschriebenen Technologien.
Um ein sinnvolles kontinuierliches Arbeiten jedes Teammitglieds zu gewährleisten, einigte sich das Team darauf ein ständig gefülltes Ticket-Pool im Repository zu führen, sodass jedes Teammitglied jederzeit selbstständig eine Aufgabe finden und bearbeiten konnte.
\subsubsection{Product Increment}
Insgesamt kam am Ende des Milestones nicht im klassischen Sinne ein Product Increment zustande, vielmehr wurden die Grundbausteine für die Entwicklung und das Teamwork gelegt.
Es wurde sich auf das bereits erwähnte Technologie-Stack entschieden sowie als Hardware ein Raspberry Pi gewählt.
Für die Entwicklung wurde ein Mono-Repository mit lerna\footnote{https://lerna.js.org/} aufgesetzt, welches Front- und Backend zusammenfasste.
Durch die Open Source Natur dieses Projektes wurde eine verständliche und hilfreiche Dokumentation für das Projekt  nötig.
So wurde auch das Repository mit einer einleitenden Readme, einem \glqqHow to Contribute\grqq Guide und einem gefüllten Backlog versehen.
\subsubsection{Retrospektive}
Nach dem ersten Meilenstein fand noch keine nennenswerte Retrospektive statt.

\subsection{Meilenstein 2}
Der zweite Meilenstein wurde für den Zeitraum 11.11.2020 bis 25.11.2020 angelegt.
Das Team sollte sich hier die ersten Grundbausteine für das Projekt legen.
\subsubsection{Ziel}
Damit das Team ein generelles Verständnis des Technologie-Stacks entwickeln konnte, musste jedes Teammitglied eine \glqqHello-World\grqq Übungsaufgabe absolvieren. Diese beinhaltete im Backend einen Feathers-Service für ein virtuelles Haustier anzulegen und diese im Frontend darzustellen.
Darüber hinaus sollte sich mit dem Raspberry Pi und dem NFC-Reader auseinandergesetzt werden, sodass NFC-Tags ausgelesen werden können.
Damit die Entwicklung zeitnah starten konnte, sollten die ersten Mockups für das User Interface erstellt werden sowie die Grundstruktur für den Bericht zu dem Projekt angefertigt werden, damit dieser begleitend schon gefüllt werden konnte.
\subsubsection{Probleme}
Zu Beginn ging der Kundenwunsch, dass bereits NFC-Tags ausgelesen und angezeigt werden können, unter und führte \colorbox{yellow}{zu Problemen mit der Zeit}.
Abgesehen davon war nur die Einarbeitung in die genutzten Technologien für manche schwieriger.
\subsubsection{Lösungen}
Bereits von Anfang an zeigte sich ein sehr guter Zusammehalt im Team und bei Problemen wurde sich schnell untereinander im Team geholfen.
Um den Kundenwunsch noch sinnvoll umzusetzen, wurden gegen Ende des Sprints Überstunden geleistet.
\subsubsection{Product Increment}
Das Produkt wurde hierbei sinnvoll um ein erstes User Interface erweitert, welches die Tag-ID vom angelegten NFC-Tags anzeigen konnte.
Der NFC-Reader wurde mit dem Raspberri Pi verbunden und als ein auslieferbares Docker Image bereitgestellt.
Auch wurde für die Entwicklung automatisierte Tests erstellt, welche neuen Code auf etablierte Code-Konventionen und Lauffähigkeit überprüften.
\subsubsection{Retrospektive}
Vor allem der Einsatz und das Know-How vom Teammitglied Anton wurde hier wertgeschätzt, welcher sich um die Automatisierungen von Tests und Deployment kümmerte.
Auch das Gruppenklima wurde vom Team sehr gelobt, die Zusammenarbeit, Kommunikation und die regelmäßigen Meetings waren liefen sehr gut.
Neu gelernt hat das Team bei diesem Meilenstein vor allem wie Vue, Feathers und Tailwind funktionieren, sowie kamen auch viele das erste mal Automatisierungsprozessen, Code- und Pull Request Reviews in Kontakt.
Das Team bemängelte bei dem Meilenstein aber auch, dass der Kundenwunsch zu spät behandelt wurde und die Prioritäten falsch gelegt wurden.
Somit wurde als Verbesserungen für die nächsten Meilensteine eine bessere Koordination und Aufgabenverteilung gewünscht.

\subsection{Meilenstein 3}
Beginenn vom 25.11.2020 bis zum 17.12.2020 wurde Meilenstein drei angelegt.
Die jeweiligen Anwendungsszenarien und Mockups sollten in diesem Sprint finalisiert werden.
Der Kunde hatte zusätzlich auch den Wunsch gestellt, dass beim Einlesen eines NFC-Tags die entsprechend hinterlegte Musik abgespielt wird.
\subsubsection{Ziel}
Das Primäre Ziel hier war vor allem den Kundenwunsch umzusetzen. Somit musste im Backend NFC-Tags mit der jeweiligen Musik angelegt werden.
Auch musste sich damit beschäftigt werden, wie Nutzer mit der Anwendung interagieren könnten, und wie das Team die Anwendung so konzipieren könnte, dass die Usablity am besten ist.
Daher sollten auch Sequenzdiagramme und weitere Mockups für das User Interface erstellt werden.
Damit jedes Teammitglied auch ohne Hardware arbeiten konnte, sollte es auch möglich sein, das Einlesen von NFC-Tags zu emulieren.
\subsubsection{Probleme}
Teilweise mussten Teammitglieder auf die Vollendung anderer Tickets abwarten, um mit neuen Tickets zu beginnen.
Ansonsten sind nur kleinere Problematiken aufgetreten, vor allem bei den Design Mockups.
Bedingt dadurch, dass das Team aus reinen Informatikern bestand, gab es nicht viel Erfahrung mit Design und Usability.
\subsubsection{Lösungen}
Macio unterstützte das Team, indem sie einen ihrer Designer bereitstellten.
Dieser half, die bestehenden Mockups zu verbessern.
Das Erstellen von und Auseinandersetzen mit Sequenzdiagrammen half bei dem Team dabei, besser das Verhalten von einem Nutzer einzuschätzen und entsprechend das User Interface zu konzipieren.
\subsubsection{Product Increment}
Zusätzlich zu dem Wunsch, dass beim anlegen eines NFC-Tags auch die jeweilig hinterlegte Spotify Musik abgespielt wurde, war auch das Einloggen über einen bestehenden Spotify Account möglich.
Auf der Startseite wurden auch die NFC-Tags entsprechend der zugeordneten Gruppe gruppiert.
Jegliche Neuerung auf dem Master wurden nun auch auf Github.io veröffentlicht.

\subsubsection{Retrospektive}
Deutlich besser lief in diesem Sprint die Problembewältigung durch Pair Programming, welches weiter die Teamarbeit und gegenseitige Hilfe verbesserte.
Auch der Workflow mit Github, den Pull Request sowie Code Reviews wurde mehr verinnerlicht.
Das Team lernte in diesem Sprint vor allem Vue Hooks kennen, und die Stärken von Feathers in Verbindung mit OAuth.
Weiterhin waren die Tickets stärker voneinander abhängig, weshalb manche Arbeiten nicht direkt möglich machte.
Um zukünftige Probleme gar nicht erst entstehen zu lassen, hatte sich das Team vorgenommen, entstandene Probleme nicht erst im Standup, sondern früher zu kommunizieren.
Genauso sollte neu Gelerntes besser kommuniziert werden, um die Arbeit effizienter zu gestalten.

\subsection{Meilenstein 4}
Meilenstein vier lag innerhalb der Weihnachtszeit und wurde deswegen mit einem größeren Zeitraum vom 17.12.2020 bis 14.01.2021 geplant.
Mit diesem Meilenstein begann die funktionale und visuelle Umsetzung der Hauptfunktionen.
\subsubsection{Ziel}
Wie auch beim vorherigen Meilenstein war wieder hier der Fokus auf den Kundenwunsch, dass die ersten Funktionalitäten umgesetzt sind.
Dass umfasste das Anlegen von Tags, also dem Scannen von Tags, anlegen eines eigenen Namens, auswählen von hinterlegter Musik und einem eigenen Bild, sowie das bearbeiten bereits angelegter Tags.
Auch sollte für den nächsten Schritt die Mockups erstellt werden, wie ein Nutzer den Lautsprecher zur Wiedergabe auswählt sowie seine eigene Leek Box initial einrichtet.
\subsubsection{Probleme}
Zu diesem Meilenstein sind vermehrt größere Probleme aufgetreten.
Die Planung und Einschätzung der umsetzbaren Tickets wurde schlecht eingeschätzt, sowie wurde ein schlechter Fokus auf die Tickets gesetzt, die mindestens benötigt wären.
Das führte dazu, dass am Ende viele Tickets halbfertig und/oder nur schlecht getestet implementiert waren.
Über die Weihnachtszeit ging auch die Produktivität und Kommunikation im Team runter.
\subsubsection{Lösungen}
Kurzfristig wurde eine Team interne Evaluation der einzelnen Mitglieder durchgeführt.
Das half einerseits dabei, dass Probleme gelöst und die Teammotivation sowie die Produktivität wieder stieg, als auch das generelle Teamklima gebessert wurde.
\subsubsection{Product Increment}
Das Produkt wurde um die Grundfunktionen erweitert.
Es ist nun möglich, Tags anzulegen und entsprechend für diese Namen, Musik und ein Bild einzustellen.
Genauso ist es dem Nutzer nun möglich, alle angelegten Tags zu filtern.
Es wurde weiterhin auch das Design Mockup vom Macio gestellten Designer in Vue umgesetzt.
Gleichzeitig wurden hierbei auch für eine leichtere Weiterarbeit jegliche Komponenten als möglichst allgemeine Vue Komponenten angelegt, um ein einheitliches Design zu bewahren.
Für die Entwickler wurden auch die Github Pull Request Tests um Typescript Support erweitert.
\subsubsection{Retrospektive}
Besonders gut gefiel dem Team mal wieder das Teamklima, vor allem des ehrliche Miteinander.
Trotz Komplikationen und Probleme konnte am Ende dem Kunden ein schönes Product Increment geliefert werden.
Pull Requests und Tickets wurden besser beschrieben, so die Probleme besser verständlicher waren.
Vor allem durch das interne Team Review wurden Spannungen gelöst.
Das Team konnte weiterhin das Wissen mit Vue ausbauen, vor allem zu Reaktivität, Events und Kommunikation zwischen Komponenten.
Groß zu bemängeln gab es zum Scope des Meilensteins.
Das Zeitmanagement war an dieser Stelle nicht gut, das Team hatte sich zu viel vorgenommen und am Ende zu wenig Zeit gehabt.
Für eine bessere Planung sowie Abstimmung, was für die nächsten Meilensteine anfällt, hatte sich das Team geeinigt.
Auch sollte die interne Deadline früher gesetzt werden, damit nicht kurz vor dem Kundenmeeting noch viele Fehler behoben werden müssen.
Um insgesamt auch die Arbeit kontinuierlicher zu gestalten, sollten Aufgaben in kleinere Probleme aufgeteilt werden

\subsection{Meilenstein 5}
Mit Meilenstein fünf sollte innerhalb vom 14.01.2021 bis zum 28.01.2021 dafür dienen, die Entwicklung am Produkt zu beenden, sodass sich das Team mehr auf den Bericht fokussieren konnte.
\subsubsection{Ziel}
Bedingt dadurch, dass das Projekt sich dem Ende neigt, sollten in diesem Meilenstein die letzten Anwendungszenarien final umgesetzt werden, sodass die letzten zwei Meilensteine zum korrigieren der letzten kleinen Fehler genutzt werden kann.
Somit war das Ziel die Umsetzung der Einstellungen für die Leek Box selbst als auch der einzelnen Tags.
Der Code sollte aufgeräumt und das User Interface \colorbox{yellow}{einheitlich gemacht} werden.
Auch sollte die Dokumentation zur Bedienung der Leek Box stehen und die Arbeit für den Bericht beginnen.
\subsubsection{Probleme}
Wie beim vorherigen Meilenstein hat auch hier das Zeitmanagement nicht komplett funktioniert.
Gegen Ende fehlte es an Zeit, um neue umgesetzte Tickets auf Fehler zu testen und entsprechend fehlerfrei in den Master zu mergen
\subsubsection{Lösungen}
Die einzige mögliche Lösung für das Team zu diesem Problem war am Ende dem Kunden weniger neue Features vorzustellen als geplant.
\subsubsection{Product Increment}
Die Leek Box Einstellungen sind implementiert worden, über die der Nutzer den Wiedergabelautsprecher auswählen kann, die jeweilige Leek Box verwalten kann und sich auch ausloggen kann.
Auch würde ein Setup Guide angefertigt, sodass ein Nutzer selbstständig eine eigene Leek Box einrichten könnte.
Bei dem Design wurden alle Umstimmigkeiten noch beseitigt.
\subsubsection{Retrospektive}
Insgesamt hat das Team sich an dieser Stelle vor allem über das sehr positive Feedback des Kunden gefreut.
Auch ist die Produktivität des Teams wieder gut angestiegen.
Das Team konnte auch weitere Vue Funktionen kennenlernen.
Problematisch war es auch, dass manche Tickets von mehreren gleichzeitig oder nacheinander bearbeitet wurden, sodass Arbeit doppelt erledigt wurde.
Die Kommunikation von größeren Änderung an jedes Teammitglied erfolgte nur lückenhaft.
Die Punkte für Verbesserungen waren wieder das Zeitmanagement und eine bessere Kommunikation.

\subsection{Meilenstein 6}
Mit dem Meilenstein sechs vom 28.01.2021 bis zum 15.02.2021 sollte die Entwicklung beendet werden und der Fokus komplett auf den Bericht gelegt werden.
\subsubsection{Ziel}
Als vorletzter Meilenstein sollte nur noch die notwendigste Entwicklungsarbeit vollendet werden und der Fokus auf die vollendung des Berichtes werden, sodass der nächste und letzte Meilenstein für die letzten Korrekturen im Bericht genutzt werden kann.
Auch sollte noch einmal das User Interface auf einheitliches Design überprüft werden.
\subsubsection{Probleme}
\colorbox{red}{TODO:}schlauchiger start in den report

\subsubsection{Lösungen}
\colorbox{red}{TODO:}Welche Lösungen haben wir gefunden?

\subsubsection{Product Increment}
Als letzte Änderungen am Produkt hat nun der Nutzer die Möglichkeit, verschiedene Leek Boxen auszuwählen bzw. neu eingerichtete Leek Boxen mit seinem Konto zu verbinden.
Die letzten Bugfixes wurden umgesetzt sowie wurden die letzten Details aus dem Design Mockup übernommen.
\subsubsection{Retrospektive}
\colorbox{red}{TODO:}Was haben wir dabei gelernt? Neue Erkentnisse? Neue Sichtweisen?
Was lief gut, neu gelernt, was lief nicht so gut, was verbessern?

\subsection{Meilenstein 7}
Der finale Meilenstein sieben vom 15.02.2021 bis zum 05.03.2021 sollte nur noch die letzten Korrekturen des Berichtes und Vorbereitung der Finalen Abgabe beinhalten.
\colorbox{red}{TODO:}Kurze Einleitung, von, bis
\subsubsection{Ziel}
\colorbox{red}{TODO:}was haben wir uns vorgenommen, was war das ziel, was wollten wir schaffen?
\subsubsection{Probleme}
\colorbox{red}{TODO:}welche prrobleme sind aufgetreten?

\subsubsection{Lösungen}
\colorbox{red}{TODO:}Welche Lösungen haben wir gefunden?

\subsubsection{Product Increment}
\colorbox{red}{TODO:}Was ist am Ende dabei rumgekommen?

\subsubsection{Retrospektive}
\colorbox{red}{TODO:}Was haben wir dabei gelernt? Neue Erkentnisse? Neue Sichtweisen?
Was lief gut, neu gelernt, was lief nicht so gut, was verbessern?

\section{Erkenntnisse}

\section{Code Walkthrough}
\colorbox{red}{TODO:}vielleicht interessante Code Snippets?

\section{Testing}
\colorbox{red}{TODO:}wie haben wir getestet, haben wir getestet?

\section{Technische Diagramme}
\colorbox{red}{TODO:}ER Diagramme, UML, solcher krams

\newpage
\section{Anhang}
\colorbox{red}{BEISPIEL, DELETE THIS} Buchreferenz \cite{Literaturbeispiel:tom} oder Seitenref \cite{google}
\printbibliography
\end{document}
