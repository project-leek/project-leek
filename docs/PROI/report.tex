\documentclass[10pt, a4paper, draft]{article}
 \usepackage[ngerman]{babel}
 \usepackage[utf8]{inputenc}
 \usepackage[T1]{fontenc}
 \usepackage{color}
 \usepackage{hyperref}

\title{Projektbreicht Smart Music Player}
\author{Anton Bracke\\Jan Eberlein\\Tom Calvin Haak\\Julian Hahn\\Nick Loewecke}

\begin{document}
\maketitle
\tableofcontents

\section{Einleitung}
\subsection{Unternehmen}
\colorbox{red}{was macht macio aus}
\subsection{Projekt}
Im Rahmen des Bachelor-Moduls Projekt Informatik möchte Macio deren Portfolio im IoT-Bereich erweitern sowie den Empfangsraum in Kiel verschönern.
Smarte Spielzeuge mit NFC-Funktion gibt es im Kommerzbereich viele, daher soll dieses Projekt eine Open-Source-Alternative dafür schaffen.
Genauer handelt es sich um einen Smarten Music Player, der NFC sowie Spotify Connect unterstützt.
Dieser Music Player hat einen Microcontroller mit NFC-Reader.
Auf die Box können durch kleine Figuren in Form von z.B. Figuren mit dem integrierten NFC-Tag Events getriggert werden.
Bei einem Event wird vom Microcontroller über jeweilige Musikdienst-API auf einem verbundenen Gerät Musik abgespielt.
Die NFC-Tags und zugehörigen Events sollen über ein Web-Frontend konfiguriert werden können.
Dieses Web-Frontend wird von der Box ausgeliefert und soll für primär eine Smartphone bedienung gestaltet werden.
Da es sich um ein Open Source Projekt handelt mit entsprechender Lizenz, muss auch eine aussagekräftige, öffentliche Dokumentation verfasst werden.
Macio stellt die benötigte Hardware und ist für Technische Fragen unterstützend verfügbar.

\subsubsection{Minimal Requirements}
\begin{enumerate}
  \item NFC-Tags lesen, schreiben und entschlüsseln
  \item Mit Spotify Connect verbinden und arbeiten
  \item Responsive UI konzeptionieren und umsetzen
  \item Aussagekräftige Dokumentation mit Benutzerhandbuch
\end{enumerate}
\subsubsection{Stretch Goals}
\begin{enumerate}
  \item Sound Wiedergabe auf der Box selbst
  \item Unterstützung anderer Musikdienste / Plugin-Subsystem
  \item 3D-Modellierung und Print einer passenden Box
  \item Cloud-Anbindung der Box, Auslieferung des UI aus der Cloud
\end{enumerate}

\section{Machbarkeitsstudie}

\section{Design Mockups}
\colorbox{red}{setze pdfs ein}

\section{Durchführung}
\subsubsection{Technologien und Hilfsmittel}
\colorbox{red}{Vue, vscode, devops krams, etc}
Entwickelt wird mit Visual Studio Code, da dort einfach ein Subsystem mit Linux genutzt werden kann \footnote{https://code.visualstudio.com/docs/remote/wsl}.

\subsubsection{Projekt Management}
\colorbox{red}{ticket pool in github, alles zentral}

\subsubsection{Deployment Cycle}
\colorbox{red}{ziehe ticket > assigne dir selbst > draft PR > wenn fertig, setzte "undraft" > assigne 2 reviewer > merge master}

\subsubsection{Probleme während der Durchführung}
\colorbox{red}{zb }
\section{Code Walkthrough}
\colorbox{red}{vielleicht interessante Code Snippets?}
\section{Testing}
\colorbox{red}{wie haben wir getestet, haben wir getestet?}
\section{Technische Diagramme}
\colorbox{red}{ER Diagramme, UML, solcher krams}
 \end{document}